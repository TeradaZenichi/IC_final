\section{Modelagem do problema}

O problema de RSD envolve restrições não lineares e variáveis quadráticas, a fim de, em um primeiro momento, realizar uma tentativa de linearização e simplificação do problema, propõe-se uma mudança de variável que será aplicado nas equações ao longo deste documento.

\begin{align}
    I_{ij}^{sqr} = I_{ij}^{2}\;\forall ij \in \Omega_l \text{ e } V_{i}^{sqr} = V_{i}^{2}\; \forall i\in\Omega_b 
    \label{eq:change_variable}
\end{align}

Onde $I_{ij}$ é a corrente entre os nós $i$ e $j$ e $\Omega_l$ é o conjunto de circuitos do sistema.
O conjunto de circuitos corresponde a todos os ramos do sistema de distribuição com grandezas físicas correspondentes tais como impedância e corrente máxima, por exemplo.

$V_i$ é a tensão no nó $i$ e $\Omega_b$ é o conjunto de nós do sistema.
O conjunto de nós do sistema corresponde a todas as intersecções entre ramos do SDEE.

\subsection{Função objetivo}

Dado o modelo de otimização a ser interpretado pela linguagem de modelagem, é possível definir a função objetivo com base na figura~\ref{fig:SDR}.
A fim de reduzir as perdas ôhmicas na rede elétrica, a função objetivo do problema consiste em minimizar a somatória das perdas por resistência elétrica no conjunto de circuitos do sistema.

Definindo $c^{lss}$ como parâmetro que representa o custo das perdas de potência ativa na rede têm-se:

\begin{equation}
    \begin{split}
        \text{Min} = & c^{lss}\sum_{ij\in\Omega_{l}}R_{ij}I_{ij}^{2}\\
        = & c^{lss}\sum_{ij\in\Omega_{l}}R_{ij}I_{ij}^{sqr}
    \end{split}
    \label{eq:funcobjetivo}
\end{equation}
