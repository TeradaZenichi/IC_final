\section{Programação em Julia}

Há diversas vantagens em utilizar-se uma linguagem de programação como a linguagem Julia.
Uma dessas vantagens é o fato de ser gratuita, outra vantagem é a possibilidade de manipular o resultado obtendo gráficos como os que foram exibidos na seção de resultados.

O programa desenvolvido recebe estruturas de dados denominadas \emph{Dataframes}, os quais são gerados pelo programa \verb|readnetwork.jl|.

Os comandos a seguir são responsáveis por importar as bibliotecas necessárias e importar o arquivo responsável por gerar as estruturas de dados que serão usadas no problema.

\begin{lstlisting}[language = Julia]
using JuMP
using CSV
using DataFrames
using GLPK
using Blink
using PlotlyJS

#45nodes_multidemand.csv
#45nodes_network.csv
include("readnetwork.jl")
using Main.readnetwork

\end{lstlisting}

A seguir, a sequência de códigos mostra  o método usado para gerar os valores máximos dos blocos de discretização do modelo linear, bem como a inclinação dos segmentos de reta que linearizam as funções quadráticas.

\begin{lstlisting}[language = Julia,firstnumber=12]
#Parameters for PLIM formulation
mutable struct parameter
    R::Int64
    Vnom::Float64
    Vmin::Float64
    Vmax::Float64
end
plim  = parameter(50,param.Vnom[1],param.Vnom[1]*0.93,param.Vnom[1]*1.05)
dSmax = zeros(length(Ωl.num));
dSmax = plim.Vmax*Ωl.Imax/plim.R;
mS    = zeros(length(Ωl.num),plim.R);
for ol in eachrow(Ωl)
    for r in 1:plim.R
        mS[ol.num,r] = (2*r-1)*dSmax[ol.num]
    end
end
\end{lstlisting}

Comandos utilizados para definir um modelo usando o \emph{solver} \emph{GLPK} e definição das variáveis do problema de otimização.

\begin{lstlisting}[language = Julia, firstnumber = 28]
#Create a DSR optimization problem
PLIM = Model(with_optimizer(GLPK.Optimizer));
#Variables declaration
@variable(PLIM, Vqdr[Ωb.i,Ωd.num] >= 0);
@variable(PLIM, Iqdr[Ωl.num,Ωd.num] >= 0);
@variable(PLIM, PS[Ωb.i,Ωd.num]);
@variable(PLIM, QS[Ωb.i,Ωd.num]);
@variable(PLIM, Pch[Ωch.num,Ωd.num]);
@variable(PLIM, Qch[Ωch.num,Ωd.num]);
@variable(PLIM, P[Ωl.num,Ωd.num]);
@variable(PLIM, Q[Ωl.num,Ωd.num]);
@variable(PLIM, Pp[Ωl.num,Ωd.num]>=0);
@variable(PLIM, Pn[Ωl.num,Ωd.num]>=0);
@variable(PLIM, Qp[Ωl.num,Ωd.num]>=0);
@variable(PLIM, Qn[Ωl.num,Ωd.num]>=0);
@variable(PLIM, dP[Ωl.num,1:plim.R,Ωd.num]>=0);
@variable(PLIM, dQ[Ωl.num,1:plim.R,Ωd.num]>=0);
@variable(PLIM, w[Ωch.num], Bin);
\end{lstlisting}

Definição da função objetivo para redução da demanda total e fixação da barra de geração com o valor máximo de tensão bem como fixando em zero a potência de geração para as barras não geradoras.

\begin{lstlisting}[language = Julia, firstnumber = 46]
#Objective Function
@objective(PLIM, Min, cls * sum(od.T*sum(Iqdr[ol.num,od.num]*ol.R 
           for ol in eachrow(Ωl)) for od in eachrow(Ωd)));

#Define generator bar  
for od in eachrow(Ωd)    
    for ob in eachrow(Ωb)
        if ob.Tb == 1
            @constraint(PLIM, Vqdr[ob.i,od.num] == plim.Vmax^2);
        else
            @constraint(PLIM, PS[ob.i,od.num] == 0);
            @constraint(PLIM, QS[ob.i,od.num] == 0);
        end
    end
end
\end{lstlisting}

Restrição de balanço de potência para multiplas demandas.

\begin{lstlisting}[language = Julia, firstnumber = 61]
#Power balance
for od in eachrow(Ωd)    
    for ob in eachrow(Ωb)
        @constraint(PLIM, sum(P[i,od.num] for i in lines[ob.i].in) 
            - sum(P[i,od.num] + Ωl.R[i]*Iqdr[i,od.num] 
            for i in lines[ob.i].out)
            + sum(Pch[i,od.num] for i in key[ob.i].in)
            - sum(Pch[i,od.num] for i in key[ob.i].out)
            + PS[ob.i,od.num] == Ωb[!,od.P][ob.i])

        @constraint(PLIM, sum(Q[i,od.num] for i in lines[ob.i].in) 
            - sum(Q[i,od.num] + Ωl.X[i]*Iqdr[i,od.num] 
            for i in lines[ob.i].out)
            + sum(Qch[i,od.num] for i in key[ob.i].in)
            - sum(Qch[i,od.num] for i in key[ob.i].out)
            + QS[ob.i,od.num] == Ωb[!,od.Q][ob.i])
    end
end
\end{lstlisting}

Restrição de tensão entre duas barras, para diversos níveis de demanda.

\begin{lstlisting}[language = Julia, firstnumber = 77]
#Voltage between two bars
for od in eachrow(Ωd)
    for ol in eachrow(Ωl)
        @constraint(PLIM, Vqdr[ol.i,od.num] - 2(ol.R*P[ol.num,od.num] 
        + ol.X*Q[ol.num,od.num])- (ol.R^2 + ol.X^2)*Iqdr[ol.num,od.num]
        - Vqdr[ol.j,od.num] == 0)
    end
end
\end{lstlisting}

Fluxo de potência entre barras aplicas à técnica de linearização.

\begin{lstlisting}[language = Julia, firstnumber = 84]
for od in eachrow(Ωd)    
    for ol in eachrow(Ωl)
        @constraint(PLIM,plim.Vmax^2*Iqdr[ol.num,od.num] == 
            sum(dP[ol.num,r,od.num]*mS[ol.num,r] for r = 1:plim.R) 
            + sum(dQ[ol.num,r,od.num]*mS[ol.num,r] for r = 1:plim.R)) 
        @constraint(PLIM, Pp[ol.num,od.num] - Pn[ol.num,od.num] 
            == P[ol.num,od.num])
        @constraint(PLIM, Qp[ol.num,od.num] - Qn[ol.num,od.num] 
            == Q[ol.num,od.num])
        @constraint(PLIM, Pp[ol.num,od.num] + Pn[ol.num,od.num] 
            == sum(dP[ol.num,r,od.num] for r =1:plim.R))
        @constraint(PLIM, Qp[ol.num,od.num] + Qn[ol.num,od.num] 
            == sum(dQ[ol.num,r,od.num] for r =1:plim.R))
        for r=1:plim.R
            @constraint(PLIM, dP[ol.num,r,od.num] <= dSmax[ol.num])
            @constraint(PLIM, dQ[ol.num,r,od.num] <= dSmax[ol.num])
        end
    end
end
\end{lstlisting}

Aplicação da modelagem das chaves nos sistemas de interconexão.

\begin{lstlisting}[language = Julia, firstnumber = 103]
#Keys constraints
for od in eachrow(Ωd)    
    for och in eachrow(Ωch)
        @constraint(PLIM,  (plim.Vmin^2 - plim.Vmax^2)*(1-w[och.num]) 
                    <= Vqdr[och.i,od.num] - Vqdr[och.j,od.num])
        @constraint(PLIM,  Vqdr[och.i,od.num] - Vqdr[och.j,od.num] 
                    <= (plim.Vmax^2 - plim.Vmin^2)*(1-w[och.num]))
        @constraint(PLIM, -(plim.Vmax*och.Imaxch)*w[och.num] 
                    <= Pch[och.num,od.num])
        @constraint(PLIM, Pch[och.num,od.num] 
                    <= (plim.Vmax*och.Imaxch)*w[och.num])
        @constraint(PLIM, -(plim.Vmax*och.Imaxch)*w[och.num] 
                    <= Qch[och.num,od.num])
        @constraint(PLIM, Qch[och.num,od.num] 
                    <= (plim.Vmax*och.Imaxch)*w[och.num])
    end
end
\end{lstlisting}

O bloco de códigos a seguir mostra a aplicação da restrição de radialidade, limites operativos e o comando para solucionar o problema com base nas modelagens jpa descritas.

\begin{lstlisting}[language = Julia, firstnumber = 120]
#Radial constraint
@constraint(PLIM, size(Ωl,1) + sum(w[och.num] for och in eachrow(Ωch)) 
            == size(Ωb,1) - sum(Ωb.Tb));
#Voltage limits
for od in eachrow(Ωd)
    for ob in eachrow(Ωb)
        @constraint(PLIM, Vqdr[ob.i,od.num] >= plim.Vmin^2)
        @constraint(PLIM, Vqdr[ob.i,od.num] <= plim.Vmax^2)
    end
end
#Current limits
for od in eachrow(Ωd)
    for ol in eachrow(Ωl)
        @constraint(PLIM, Iqdr[ol.num,od.num] <= ol.Imax^2)
    end
end
JuMP.optimize!(PLIM)
\end{lstlisting}

O conjunto de comandos a seguir corresponde a um conjunto de aplicações usadas para exibir gráficos e estrutura de dados com os resultados obtidos.

\begin{lstlisting}[language = Julia, firstnumber = 137]
data = GenericTrace[]
for od in eachrow(Ωd)
    #push!(data,scatter(;x=Ωb.i,y=Ωb[!,od.P],mode="lines+markers",
            name=string("demand level number ", od.num)))
    push!(data,bar(;x=Ωb.i,y=Ωb[!,od.P],
            name=string("demand level number ", od.num)))
end
display(plot(data,Layout(title="Active power demand", 
            xaxis_title="Bars", yaxis_title="Active power [kW]")))
data = GenericTrace[]
for od in eachrow(Ωd)
    #push!(data,scatter(;x=Ωb.i,y=Ωb[!,od.Q],mode="lines+markers",
            name=string("demand level number ", od.num)))
    push!(data,bar(;x=Ωb.i,y=Ωb[!,od.Q],name=
            string("demand level number ",od.num)))
end
display(plot(data,Layout(title="Reactive power demand", 
            xaxis_title="Bars", yaxis_title="Reactive power [kVar]")))

#Results
obj_value_plim = JuMP.objective_value(PLIM);
Vlim  = zeros(length(Ωb.i),length(Ωd.num));   
Ilim  = zeros(length(Ωl.num),length(Ωd.num));
PSlim = zeros(length(Ωb.i),length(Ωd.num));   
QSlim = zeros(length(Ωb.i),length(Ωd.num));
Plim  = zeros(length(Ωl.num),length(Ωd.num)); 
Qlim  = zeros(length(Ωl.num),length(Ωd.num));
Pchlim  = zeros(length(Ωch.num),length(Ωd.num)); 
Qchlim  = zeros(length(Ωch.num),length(Ωd.num));

for od in eachrow(Ωd)    
    for ob in eachrow(Ωb)
        Vlim[ob.i,od.num] = sqrt(JuMP.value(Vqdr[ob.i,od.num]))
        PSlim[ob.i,od.num] = JuMP.value(PS[ob.i,od.num])
        QSlim[ob.i,od.num] = JuMP.value(QS[ob.i,od.num])
    end
end
loss_plim = Array{Float64,1}(undef,length(Ωd.num));
for od in eachrow(Ωd)
    demand_plim = 0;
    for ol in eachrow(Ωl)
        Ilim[ol.num,od.num] = sqrt(JuMP.value(Iqdr[ol.num,od.num]))
        Plim[ol.num,od.num] = JuMP.value(P[ol.num,od.num])
        Qlim[ol.num,od.num] = JuMP.value(Q[ol.num,od.num])
        demand_plim = demand_plim + ol.R*Ilim[ol.num,od.num]^2
    end
    loss_plim[od.num] = demand_plim*od.T
    for och in eachrow(Ωch)
        Pchlim[och.num,od.num] = JuMP.value(Pch[och.num,od.num])
        Qchlim[och.num,od.num] = JuMP.value(Qch[och.num,od.num])
    end

end
wlim = Int64[];
for och in eachrow(Ωch)
    push!(wlim,JuMP.value(w[och.num]))
end
vmax_plot = scatter(;x=Ωb.i,y=ones(length(Ωb.i))*plim.Vmax,
            name="Max voltage");
vmin_plot = scatter(;x=Ωb.i,y=ones(length(Ωb.i))*plim.Vmin,
            name="Min voltage");
voltages = [vmax_plot,vmin_plot]
for od in eachrow(Ωd)
    data = scatter(;x=Ωb.i,y=Vlim[:,od.num[1]],name= 
    string("bars voltage of demand number ", od.num),
    mode="lines+markers")
    push!(voltages,data)
end
display(plot(voltages,Layout(title="Network voltages for 
        all demands after reconfiguration", xaxis_title = 
        "Bars",yaxis_title = "Voltage [kV]")))
\end{lstlisting}

Aplicação da topologia inicial e módulos para exibição do resultado inicial e resultados gerais.

\begin{lstlisting}[language = Julia, firstnumber = 208]
#Create a DSR optimization problem
Init = Model(with_optimizer(GLPK.Optimizer));
#Variables declaration
@variable(Init, Vqdr[Ωb.i,Ωd.num] >= 0);
@variable(Init, Iqdr[Ωl.num,Ωd.num] >= 0);
@variable(Init, PS[Ωb.i,Ωd.num]);
@variable(Init, QS[Ωb.i,Ωd.num]);
@variable(Init, Pch[Ωch.num,Ωd.num]);
@variable(Init, Qch[Ωch.num,Ωd.num]);
@variable(Init, P[Ωl.num,Ωd.num]);
@variable(Init, Q[Ωl.num,Ωd.num]);
@variable(Init, Pp[Ωl.num,Ωd.num]>=0);
@variable(Init, Pn[Ωl.num,Ωd.num]>=0);
@variable(Init, Qp[Ωl.num,Ωd.num]>=0);
@variable(Init, Qn[Ωl.num,Ωd.num]>=0);
@variable(Init, dP[Ωl.num,1:plim.R,Ωd.num]>=0);
@variable(Init, dQ[Ωl.num,1:plim.R,Ωd.num]>=0);
@variable(Init, w[Ωch.num], Bin);

#Objective Function
@objective(Init, Min, cls * sum(od.T*sum(Iqdr[ol.num,od.num]*ol.R 
            for ol in eachrow(Ωl)) for od in eachrow(Ωd)));

#Define generator bar  
for od in eachrow(Ωd)    
    for ob in eachrow(Ωb)
        if ob.Tb == 1
            @constraint(Init, Vqdr[ob.i,od.num] == plim.Vmax^2);
        else
            @constraint(Init, PS[ob.i,od.num] == 0);
            @constraint(Init, QS[ob.i,od.num] == 0);
        end
    end
end
#Power balance
for od in eachrow(Ωd)    
    for ob in eachrow(Ωb)
        @constraint(Init, sum(P[i,od.num] for i in lines[ob.i].in) 
            - sum(P[i,od.num] + Ωl.R[i]*Iqdr[i,od.num] 
            for i in lines[ob.i].out)
            + sum(Pch[i,od.num] for i in key[ob.i].in)
            - sum(Pch[i,od.num] for i in key[ob.i].out)
            + PS[ob.i,od.num] == Ωb[!,od.P][ob.i])

        @constraint(Init, sum(Q[i,od.num] for i in lines[ob.i].in) 
            - sum(Q[i,od.num] + Ωl.X[i]*Iqdr[i,od.num] 
            for i in lines[ob.i].out)
            + sum(Qch[i,od.num] for i in key[ob.i].in)
            - sum(Qch[i,od.num] for i in key[ob.i].out)
            + QS[ob.i,od.num] == Ωb[!,od.Q][ob.i])
    end
end
#Voltage between two bars
for od in eachrow(Ωd)
    for ol in eachrow(Ωl)
        @constraint(Init, Vqdr[ol.i,od.num] - 2(ol.R*P[ol.num,od.num] 
            + ol.X*Q[ol.num,od.num])- 
            (ol.R^2 + ol.X^2)*Iqdr[ol.num,od.num] 
            - Vqdr[ol.j,od.num] == 0)
    end
end
#Piecewise linearization 
for od in eachrow(Ωd)    
    for ol in eachrow(Ωl)
        @constraint(Init,plim.Vmax^2*Iqdr[ol.num,od.num] 
            == sum(dP[ol.num,r,od.num]*mS[ol.num,r] for r = 1:plim.R) 
            + sum(dQ[ol.num,r,od.num]*mS[ol.num,r] for r = 1:plim.R)) 
        @constraint(Init, Pp[ol.num,od.num] - Pn[ol.num,od.num] 
            == P[ol.num,od.num])
        @constraint(Init, Qp[ol.num,od.num] - Qn[ol.num,od.num] 
            == Q[ol.num,od.num])
        @constraint(Init, Pp[ol.num,od.num] + Pn[ol.num,od.num] 
            == sum(dP[ol.num,r,od.num] for r =1:plim.R))
        @constraint(Init, Qp[ol.num,od.num] + Qn[ol.num,od.num] 
            == sum(dQ[ol.num,r,od.num] for r =1:plim.R))
        for r=1:plim.R
            @constraint(Init, dP[ol.num,r,od.num] <= dSmax[ol.num])
            @constraint(Init, dQ[ol.num,r,od.num] <= dSmax[ol.num])
        end
    end
end
#Keys constraints
for od in eachrow(Ωd)    
    for och in eachrow(Ωch)
        if och.ei == 0
            @constraint(Init, Pch[och.num,od.num] == 0)
            @constraint(Init, Qch[och.num,od.num] == 0)
        end
        @constraint(Init,  Vqdr[och.j,od.num] - Vqdr[och.i,od.num] 
            >= (-plim.Vmax^2)*(1-och.ei))
        @constraint(Init,  Vqdr[och.i,od.num] - Vqdr[och.j,od.num] 
            <= (plim.Vmax^2)*(1-och.ei))
    end
end
JuMP.optimize!(Init)
#Results
obj_value = JuMP.objective_value(Init);
Vinit  = zeros(length(Ωb.i),length(Ωd.num));   
Iinit  = zeros(length(Ωl.num),length(Ωd.num));
PSinit = zeros(length(Ωb.i),length(Ωd.num));   
QSinit = zeros(length(Ωb.i),length(Ωd.num));
Pinit  = zeros(length(Ωl.num),length(Ωd.num)); 
Qinit  = zeros(length(Ωl.num),length(Ωd.num));
Pchinit  = zeros(length(Ωch.num),length(Ωd.num)); 
Qchinit  = zeros(length(Ωch.num),length(Ωd.num));

for od in eachrow(Ωd)
    for ob in eachrow(Ωb)
        Vinit[ob.i,od.num] = sqrt(JuMP.value(Vqdr[ob.i,od.num]))
        PSinit[ob.i,od.num] = JuMP.value(PS[ob.i,od.num])
        QSinit[ob.i,od.num] = JuMP.value(QS[ob.i,od.num])
    end
end
loss_init = Array{Float64,1}(undef,length(Ωd.num));
for od in eachrow(Ωd)
    demand_init = 0;
    for ol in eachrow(Ωl)
        Iinit[ol.num,od.num] = sqrt(abs(JuMP.value(Iqdr[ol.num,od.num])))
        Pinit[ol.num,od.num] = JuMP.value(P[ol.num,od.num])
        Qinit[ol.num,od.num] = JuMP.value(Q[ol.num,od.num])
        demand_init = demand_init + ol.R*Iinit[ol.num,od.num]^2
    end
    loss_init[od.num] = demand_init*od.T
    for och in eachrow(Ωch)
        Pchinit[och.num,od.num] = JuMP.value(Pch[och.num,od.num])
        Qchinit[och.num,od.num] = JuMP.value(Qch[och.num,od.num])
    end
    

end
vmax_plot = scatter(;x=Ωb.i,y=ones(length(Ωb.i))*plim.Vmax,
    name="Max voltage");
vmin_plot = scatter(;x=Ωb.i,y=ones(length(Ωb.i))*plim.Vmin,
    name="Min voltage");
voltages = [vmax_plot,vmin_plot]
for od in eachrow(Ωd)
    data = scatter(;x=Ωb.i,y=Vinit[:,od.num[1]],name=
    string("bars voltage of demand number ", od.num),mode="lines+markers")
    push!(voltages,data)
end
display(plot(voltages,Layout(title="Network voltages for 
    all demands before reconfiguration",xaxis_title = 
    "Bars",yaxis_title = "Voltage [kV]")))


results = Array{DataFrame}(undef,length(Ωd.num))
for od in eachrow(Ωd)
    demand = DataFrame();
    par = ["Active power Losses [kW]", "Minimum voltage magnitude [kV]", 
        "Bar of minimum voltage magnitude",
        "Substation active power [kW]", 
        "Substation reactive power [kVar]"];
    Initial = [string(round(loss_init[od.num], digits = 2 )), 
        string(round(minimum(Vinit[:,od.num]),digits =2)),
        string(argmin(Vinit[:,od.num])), 
        string(round(maximum(PSinit[:,od.num]),digits=2)), 
        string(round(maximum(QSinit[:,od.num])))  
        ];
    
    Reconfigured = [string(round(loss_plim[od.num], digits = 2 )), 
        string(round(minimum(Vlim[:,od.num]),digits =2)),
        string(argmin(Vlim[:,od.num])), 
        string(round(maximum(PSlim[:,od.num]),digits=2)), 
        string(round(maximum(QSlim[:,od.num]),digits=2))  
        ];
    demand.Parameter = par;
    demand.Initial = Initial;
    demand.Reconfigured = Reconfigured;
    results[od.num] = demand
end

print("Cost of losses per demand period before reconfiguration: 
    US\$ ", round(JuMP.objective_value(Init),digits=2),"\n");
print("Cost of losses per demand period after  reconfiguration: 
    US\$ ", round(JuMP.objective_value(PLIM),digits=2),"\n\n");
for od in eachrow(Ωd)
    print("Demand number  ", od.num, " results:")
    print(results[od.num],"\n\n");
end

keys_states = DataFrame();
keys_states.Initial = Ωch.ei;
keys_states.Reconfigured = wlim;
print(keys_states);
\end{lstlisting}