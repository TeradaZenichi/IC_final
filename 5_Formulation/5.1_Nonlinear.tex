\subsection{Problema de programação não linear}

Com as equações já descritas nas seções anteriores é possível obter o modelo do problema de PNLIM.

O problema descrito no conjunto equações destacadas é considerado um problema de programação não linear devido à presença da restrição expressa na equação~\eqref{eq:PNLIM_power}, uma vez que ela é o produto de duas variáveis que representam tensão e corrente do sistema ($V_{j}^{sqr}$ e $I_{ij}^{sqr}$) que são iguais a soma dos quadrados das variáveis que representam as potências ativas e reativas do sistema ($P_{ij}$ e $Q_{ij}$).


\begin{tcolorbox}[breakable,pad at break*=1mm,colback=white!10,title =\textbf{Problema de PNLIM para RSD}]

\begin{equation}\label{eq:PNLIM_funcobj}
    \text{Min } c^{lss}\underset{ij\in\Omega_{l}}{\sum} R_{ij}I_{ij}^{sqr}
\end{equation}

\begin{equation}
    \sum_{ji\in\Omega_{l}}P_{ji} - \sum_{ij\in\Omega_{l}}(P_{ij} + R_{ij}I_{ij}^{sqr})+ \sum_{ji\in\Omega_{ch}}P_{ji}^{ch} -\sum_{ij\in\Omega_{ch}}P_{ij}^{ch} + P_{i}^{S} = P_{i}^{D}\quad\forall i \in\Omega_{b}\label{eq:PNLIM_fluxoP}  
\end{equation}
    
\begin{equation}
    \sum_{ji\in\Omega_{l}}Q_{ji} - \sum_{ij\in\Omega_{l}}(Q_{ij} + X_{ij}I_{ij}^{sqr})+ \sum_{ji\in\Omega_{ch}}Q_{ji}^{ch} -\sum_{ij\in\Omega_{ch}}Q_{ij}^{ch} + Q_{i}^{S} = P_{i}^{D}\quad\forall i \in\Omega_{b}
    \label{eq:PNLIM_fluxoQ}
\end{equation}

\begin{equation}\label{eq:PNLIM_voltage}
    V_{i}^{sqr} - 2(R_{ij}P_{ij} + X_{ij}Q_{ij}) - Z_{ij}^{2}I_{ij}^{2} - V_{j}^{sqr} = 0\quad\forall ij \in \Omega_{l}
\end{equation}

\begin{equation}\label{eq:PNLIM_power}
    V_{j}^{sqr}I_{ij}^{sqr} = P_{ij}^{2}+Q_{ij}^{2}\;\qquad\forall ij \in \Omega_{l}
\end{equation}

\begin{equation}\label{eq:PNLIM_voltagekeys}
    -(\overline{V}^{2} - \underline{V}^{2})(1-w_{ij}) \leq V_{i}^{sqr} - V_{j}^{sqr} \leq (\overline{V}^{2} - \underline{V}^{2})(1-w_{ij})\qquad\forall ij\in\Omega_{ch}
\end{equation}

\begin{equation}\label{eq:PNLIM_Pch}
    -(\overline{V}\,\overline{I}_{ij}^{ch})w_{ij} \leq P_{ij}^{ch} \leq (\overline{V}\,\overline{I}_{ij}^{ch})w_{ij}\qquad\forall ij\in\Omega_{ch}   
\end{equation}
    
\begin{equation}\label{eq:PNLIM_Qch}
    -(\overline{V}\,\overline{I}_{ij}^{ch})w_{ij} \leq Q_{ij}^{ch} \leq (\overline{V}\,\overline{I}_{ij}^{ch})w_{ij}\qquad\forall ij\in\Omega_{ch}   
\end{equation}

\begin{equation}\label{eq:PNLIM_radialidade}
    |\Omega_{l}| + \sum_{ij\in\Omega_{ch}}w_{ij} = |\Omega_{b}| - 1
\end{equation}

\begin{equation}\label{eq:PNLIM_voltagelim}
    \underline{V}^{2} \leq V_{i}^{sqr} \leq \overline{V}^{2}\qquad\forall i \in\Omega_{b}
\end{equation}

\begin{equation}\label{eq:PNLIM_currentlim}
    0 \leq I_{ij}^{sqr} \leq \overline{I}_{ij}^{2} \qquad\forall ij\in\Omega_{l}
\end{equation}

\begin{equation}
    w_{ij}\in\{0,1\}\qquad\forall ij\in \Omega_{l}
\end{equation}


\end{tcolorbox}


