\section{Conclusão}

Com este projeto foi possível observar como a reconfiguração de uma rede de distribuição de energia elétrica que opera de forma radial pode reduzir de forma considerável as perdas de potência ativa.
Além disso é possível observar como a reconfiguração do sistema de distribuição de energia elétrica manteve o nível de tensão em um intervalo determinado por norma em todas as barras do sistema.
Tensões maiores que o valor mínimo estipulado pela agência reguladora implica em uma diminuição da corrente que circula pelo conjunto de ramos do circuito de distribuição, essa redução da corrente aumenta a vida útil dos equipamentos implicando de forma indireta na redução de custo por reposição.

Esta parte do trabalho realizado mostrou que a metodologia para a transformação de um programa de PNLIM para um problema de PLIM substitui com bastante precisão o modelo não linear além de mostrar-se mais rápida e confiável, dado que modelos lineares são mais rápidos, robustos e sempre convergem em um ótimo global.

Por fim, com base nos resultados e em sua discussão, torna-se passível de discussão a ideia de distribuição de chaves ao longo dos sistemas de distribuição de energia elétrica para reconfiguração da mesma.