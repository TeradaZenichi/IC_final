\section{Introdução}

Os sistemas de distribuição de energia elétrica (SDEE) são planejados como redes de malhas interconectadas. Com a finalidade de operar de forma mais eficiente de modo a coordenar a proteção do sistema mais facilmente e reduzir a corrente de curto circuito, o SDEE opera com uma topologia radial \cite{Romais2014ReconfiguracaoMista}.

Os SDEE devem operar de forma a respeitar tanto as restrições de carga quanto as restrições operacionais. Dado que o sistema está operando em regime permanente é interessante operá-lo em estado de mínimas perdas. Para isso reconfigura-se o sistema de distribuição de modo a reduzir as perdas ôhmicas ao longo da rede.

O problema de reconfiguração do sistema de distribuição (RSD) é um problema de planejamento da operação das chaves alocadas ao longo dos alimentadores e consiste na abertura e/ou fechamento das chaves com o objetivo de melhorar um índice de desempenho.

A reconfiguração ótima é uma importante ferramenta para aumentar a confiabilidade de um SDEE, especialmente quando a automação avançada e tecnologias de redes inteligentes (smartgrids) tornam-se mais importantes e mais acessíveis às concessionarias de distribuição.

Os benefícios de se reduzir as perdas de potência ativa no sistema de distribuição são:% 

\begin{itemize}
    \item Alívio do sistema de distribuição: com a redução das perdas de potência ativa, o sistema é aliviado, o que leva a uma maior vida útil dos equipamentos, uma maior capacidade de fornecimento e um melhor perfil da magnitude de tensão no sistema;

    \item Adiamento de investimentos para a expansão do sistema de distribuição: a redução das perdas de potência tem como consequência a redução dos fluxos de potência nos condutores, e desta forma é adiada a necessidade de reforços na rede.
    
    \item Melhoria na qualidade de energia: a reconfiguração melhora o perfil da magnitude de tensão do sistema;
    
    \item Adiamento da necessidade de ampliação da capacidade de transmissão: a rede de distribuição pode reduzir o carregamento de linhas de transmissão no horário de pico, aumentando efetivamente a capacidade de transmissão;
    
    \item Adiamento da ampliação da capacidade de geração: menos unidades de geração operando são necessárias no horário de pico;
    
    \item Redução do uso de combustíveis: ao reduzir as perdas, reduz-se a necessidade de geração de energia a partir de fontes não renováveis, o que leva a uma economia no uso de combustíveis fósseis;
    
    \item Benefícios ambientais: a redução no uso de combustíveis fósseis tem como consequência a redução de poluição;
    
    \item Redução na contratação de energia elétrica para grandes clientes: ao reduzir as perdas das redes dos grandes clientes, reduz-se o consumo de energia elétrica.
\end{itemize}

A reconfiguração do sistema de distribuição é um problema de otimização cuja modelagem matemática pode ser classificada das seguintes formas, como descrito em \cite{Goncalves2013ModelosRadiais}:

\begin{itemize}
    \item PL: Programação Linear
    
    \item PNL: Programação não linear
    
    \item PLIM: Programação linear inteiro misto
    
    \item PNLIM: Programação não linear inteiro misto
\end{itemize}

Para solucionar esses problemas utilizam-se de ``solvers'' que são programas cuja finalidade é encontrar uma solução ótima para um problema de programação.

Problemas de PL podem ser resolvidos usando algoritmos convencionais como simplex e pontos interiores.
Já para problemas de PNL existem diversas técnicas como método de Newton e relaxação Lagrangeana.
Problemas de PLIM podem ser resolvidos usando técnicas baseadas em branch and bound.
Por fim um problema de PNLIM são complicados de serem solucionados e imagina-se que solvers comerciais usem algoritmos baseados em \emph{branch and bound}. 
O problema de RSD, como será visto, é um problema não linear inteiro misto.
Problemas não lineares são complicados de serem solucionados devido à dificuldade de convergência dos algoritmos usados para determinar soluções ótimas. 
Além disso, em problemas de PL e PLIM existem condições necessárias e suficientes de otimização teoricamente provadas que garantem se uma dada solução é factível ou não.
Já para problemas PNL e PNLIM não existem tais condições. Por outro lado existem as heurísticas e meta-heurísticas que buscam resolver esses problemas de forma a buscar uma solução satisfatória em tempo adequado \cite{Goncalves2013ModelosRadiais}.  


Alguns trabalhos relevantes na literatura que abordam a reconfiguração de redes radiais com demandas fixas foram tratados em \cite{Baran1989NetworkBalancing} abordando algoritmos heurísticos.

Algoritmos genéticos como mostrado em  \cite{Souza2015AlgoritmoVariaveis} se mostraram interessantes para resolver problemas de característica não linear.
Em \cite{deCastro2002AnOptimization} mostra-se uma interessante abordagem que simula o comportamento do sistema imunológico do corpo humano para solucionar um problema de otimização não linear.
Outra abordagem interessante é a transformação de um problema PNLIM em um problema PCSOIM (programação cônica de segunda ordem inteiro misto) através do relaxamento de uma restrição do problema \cite{Romais2014ReconfiguracaoMista}.